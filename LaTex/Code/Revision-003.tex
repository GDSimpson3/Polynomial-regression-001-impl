\documentclass{article}
\usepackage{graphicx} % Required for inserting images
\usepackage{hyperref} % For clickable links
\usepackage{amsmath} 
\usepackage{listings}
\usepackage{xcolor} % optional, for colors
\usepackage{array}

\title{Polynomial Regression Impl 001}
\author{Gavin Simpson}
\date{December 2025}

\begin{document}

\maketitle

\tableofcontents

\newpage

\section{Resources and About}

Date: \textit{13/12/2025}\newline
Revision: \textit{3}\newline
\newline
Github Repository: 
\href{https://github.com/GDSimpson3/Polynomial-regression-001-impl}{\texttt{github.com/GDSimpson3/Polynomial-regression-001-impl}}
PDF URL: 
\href{https://github.com/GDSimpson3/Polynomial-regression-001-impl/blob/main/README.pdf}{\texttt{github.com/GDSimpson3/Polynomial-regression-001-impl/README.pdf}}




\newpage

\section{Dynamic Orders}


\subsection{Dynamic Exponents 3SLOTS 6PARAM DSQuadratic}

Works on a simple Quadratic Dataset

Uses 6 Parameters:

\begin{table}[h!]
\centering
\begin{tabular}{|c|c|}
\hline
\textit{Symbol} & Meaning \\ \hline
\textit{Ca} & Coefficient A \\
\textit{Ea} & Exponent A \\
\textit{Cb} & Coefficient B \\
\textit{Eb} & Exponent B \\
\textit{Cc} & Coefficient C \\
\textit{Ec} & Exponent C \\ \hline
\end{tabular}
\caption{Mapping of symbols to coefficient and exponent terms}
\end{table}



Uses the Gradient Descent to find the best exponents and Coefficients

Currently was able to bring MSE down to \textbf{3} with \textbf{20,000} iterations


It currently uses 3 static slots in the form of

\[
Cax^{Ea} + Cbx^{Eb} + Ccx^{Ec}
\]

